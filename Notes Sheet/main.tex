\documentclass[12pt]{article}
\usepackage{amsmath}
\usepackage{amssymb}
\usepackage{booktabs}
\usepackage{hyperref}
\hypersetup{
    colorlinks=true,
    linkcolor=blue,
    filecolor=magenta,      
    urlcolor=cyan,
}
\title{Mechanics of Materials: Key Formulas, Notations, and Units}
\author{Viggo K. Hansen}
\date{\today}

\begin{document}

\maketitle

\tableofcontents

\section{\hyperref[sec:notation]{Notation and Units}}
\label{sec:notation}
\begin{tabular}{ll}
    \toprule
    Symbol & Description [Units] \\
    \midrule
    $P$ & Axial load [N] \\
    $A$ & Cross-sectional area [m$^2$] \\
    $L$ & Length of the member [m] \\
    $E$ & Young's modulus [Pa] \\
    $G$ & Shear modulus [Pa] \\
    $V$ & Shear force [N] \\
    $T$ & Torque [Nm] \\
    $J$ & Polar moment of inertia [m$^4$] \\
    $M$ & Bending moment [Nm] \\
    $I$ & Area moment of inertia [m$^4$] \\
    $\gamma$ & Shear strain [dimensionless] \\
    $\theta$ & Angle of twist [rad] \\
    $y$ & Distance from neutral axis [m] \\
    $\sigma_x, \sigma_y$ & Normal stresses in $x$ and $y$ directions [Pa] \\
    $\tau_{xy}$ & Shear stress in $xy$ plane [Pa] \\
    $\nu$ & Poisson's ratio [dimensionless] \\
    $\alpha$ & Coefficient of thermal expansion [K$^{-1}$] \\
    $\sigma_z$ & Normal stress in $z$ direction [Pa] \\
    \bottomrule
\end{tabular}

\section{Stress and Strain}
\subsection{Normal Stress}
\begin{equation}
\sigma = \frac{P}{A}
\end{equation}
This equation relates the normal stress to the applied axial load and the cross-sectional area.

\subsection{Shear Stress}
\begin{equation}
\tau = \frac{V}{A}
\end{equation}
Shear stress is calculated as the shear force divided by the area over which it acts.

\subsection{Normal Strain}
\begin{equation}
\varepsilon = \frac{\Delta L}{L}
\end{equation}
Normal strain is the ratio of change in length to the original length.

\section{Stress-Strain Relationships}
\subsection{Hooke's Law}
\begin{equation}
\sigma = E \varepsilon
\end{equation}
This linear relationship describes how stress is proportional to strain within the elastic limit.

\subsection{Shear Modulus}
\begin{equation}
\tau = G \gamma
\end{equation}
Shear stress is directly proportional to shear strain, with $G$ as the proportionality constant.

\subsection{Generalized Hooke's Law}
\begin{align}
\varepsilon_x &= \frac{1}{E} \left( \sigma_x - \nu(\sigma_y + \sigma_z) \right) \\
\gamma_{xy} &= \frac{\tau_{xy}}{G}
\end{align}
These equations account for the effect of Poisson's ratio in multi-axial stress states.

\section{Torsion}
\subsection{Torsional Stress}
\begin{equation}
\tau = \frac{T r}{J}
\end{equation}
Here, $r$ is the radial distance from the center of the shaft.

\subsection{Angle of Twist}
\begin{equation}
\theta = \frac{T L}{G J}
\end{equation}
The angle of twist is proportional to the torque and length, inversely to the shear modulus and polar moment of inertia.

\subsection{Thin-Walled Tubes (Average Shear Stress)}
\begin{equation}
\tau_{\text{avg}} = \frac{T}{2t A_m}
\end{equation}
where $t$ is the thickness of the tube wall and $A_m$ is the mean area enclosed by the centerline of the tube's cross-section.

\subsection{Power in Torsion}
\begin{equation}
P = T \omega = 2\pi f T
\end{equation}
This equation shows how power is related to torque and rotational speed or frequency.

\section{Beam Bending}
\subsection{Bending Stress}
\begin{equation}
\sigma = \frac{M y}{I}
\end{equation}
Bending stress varies linearly with the distance from the neutral axis.

\subsection{Unsymmetric Bending}
\begin{align}
\sigma &= -\frac{M_z y}{I_z} + \frac{M_y z}{I_y} \\
\tan \alpha &= \frac{I_z}{I_y} \tan \theta
\end{align}
These equations account for bending in two directions, where $\alpha$ is the angle between the neutral axis and the $x$-axis.

\subsection{Transverse Shear Stress}
\begin{equation}
\tau = \frac{VQ}{It}
\end{equation}
Here, $Q$ is the first moment of area about the neutral axis.

\subsection{Deflection (Curvature)}
\begin{equation}
\frac{d^2 y}{dx^2} = \frac{M}{E I}
\end{equation}
This differential equation describes how deflection relates to bending moment.

\section{Combined Stresses}
\subsection{Principal Stresses}
\begin{equation}
\sigma_{1,2} = \frac{\sigma_x + \sigma_y}{2} \pm \sqrt{\left(\frac{\sigma_x - \sigma_y}{2}\right)^2 + \tau_{xy}^2}
\end{equation}
Principal stresses are calculated to find the maximum and minimum normal stresses on a plane.

\subsection{Maximum Shear Stress}
\begin{equation}
\tau_{\text{max}} = \sqrt{\left(\frac{\sigma_x - \sigma_y}{2}\right)^2 + \tau_{xy}^2}
\end{equation}
This gives the maximum shear stress in a plane.

\subsection{Stress Transformation}
\begin{align}
\sigma_x' &= \frac{\sigma_x + \sigma_y}{2} + \frac{\sigma_x - \sigma_y}{2} \cos 2\theta + \tau_{xy} \sin 2\theta \\
\tau_{x'y'} &= -\frac{\sigma_x - \sigma_y}{2} \sin 2\theta + \tau_{xy} \cos 2\theta
\end{align}
These transformations help in analyzing stress on any rotated plane.

\section{Mohr's Circle}
\subsection{Center and Radius}
\begin{equation}
\text{Center} = \frac{\sigma_x + \sigma_y}{2}, \quad \text{Radius} = \sqrt{\left(\frac{\sigma_x - \sigma_y}{2}\right)^2 + \tau_{xy}^2}
\end{equation}
\textit{Note: Mohr's Circle is a graphical method used to determine stress states under transformation.}

\section{Energy Methods}
\subsection{Axial Strain Energy}
\begin{equation}
U = \frac{1}{2} \frac{P^2 L}{E A}
\end{equation}
Strain energy due to axial load.

\subsection{Torsional Strain Energy}
\begin{equation}
U = \frac{1}{2} \frac{T^2 L}{G J}
\end{equation}
Energy stored due to torsion.

\subsection{Strain Energy for Shear}
\begin{equation}
U = \int \frac{V^2}{2 G A} dx
\end{equation}
This is the strain energy due to shear forces along the length of a member.

\section{Thin-Walled Pressure Vessels}
\subsection{Cylinders}
\begin{align}
\sigma_1 &= \frac{p r}{t} \\
\sigma_2 &= \frac{p r}{2t}
\end{align}
Valid for thin-walled cylinders where the wall thickness is small compared to the radius.

\subsection{Spheres}
\begin{equation}
\sigma_1 = \sigma_2 = \frac{p r}{2t}
\end{equation}
For thin-walled spheres, both stresses are equal due to symmetry.

\section{Additional Formulas}
\subsection{Euler's Buckling Formula for Columns}
\begin{equation}
P_{cr} = \frac{\pi^2 E I}{L_e^2}
\end{equation}
where $L_e$ is the effective length of the column, accounting for end conditions.

\subsection{Thermal Expansion Stress}
\begin{equation}
\sigma = E \alpha \Delta T
\end{equation}
where $\alpha$ is the coefficient of thermal expansion, and $\Delta T$ is the change in temperature.

\subsection{Secant Formula (Advanced)}
\begin{equation}
\sigma_{\text{max}} = \frac{P}{A} \left( 1 + e \frac{r^2}{L^2} \sec \left( \frac{\pi L}{2r} \right) \right)
\end{equation}
This formula is used for columns with eccentric loading. Here, $e$ is the eccentricity of the load.

\section{Dynamic Effects}

\subsection{Impact Loading}
\begin{equation}
\sigma_{\text{impact}} = \sigma_{\text{static}} \cdot \left(1 + \sqrt{1 + \frac{2E h}{g \sigma_{\text{static}}}}\right)
\end{equation}
where $\sigma_{\text{impact}}$ is the stress under impact, $\sigma_{\text{static}}$ is the static stress, $h$ is the height from which the load is dropped, $g$ is the acceleration due to gravity, and $E$ is Young's modulus.

\subsection{Vibration and Resonance}
The natural frequency of a system can be calculated by:
\begin{equation}
f_n = \frac{1}{2\pi} \sqrt{\frac{k}{m}}
\end{equation}
where $f_n$ is the natural frequency, $k$ is the stiffness of the system, and $m$ is the mass.

\section{Fatigue Analysis}

\subsection{S-N Curve}
The fatigue life of a material can often be described by the S-N curve:
\begin{equation}
N = C \left( \frac{\sigma_a}{\sigma_f} \right)^{-b}
\end{equation}
where $N$ is the number of cycles to failure, $\sigma_a$ is the alternating stress, $\sigma_f$ is the fatigue strength coefficient, $C$ and $b$ are material constants.

\subsection{Miner's Rule for Cumulative Damage}
For variable amplitude loading, the damage accumulation can be calculated using:
\begin{equation}
\sum \frac{n_i}{N_i} = D
\end{equation}
where $n_i$ is the number of cycles at stress level $i$, $N_i$ is the number of cycles to failure at that stress level, and $D$ is the cumulative damage (failure when $D \geq 1$).

\section{Advanced Topics in Plasticity}

\subsection{Ramberg-Osgood Equation}
For materials beyond their elastic limit, the stress-strain relationship can be modeled by:
\begin{equation}
\epsilon = \frac{\sigma}{E} + \left( \frac{\sigma}{\sigma_0} \right)^n
\end{equation}
where $\epsilon$ is the total strain, $\sigma$ is the stress, $E$ is Young's modulus, $\sigma_0$ is a reference stress (often yield stress), and $n$ is the hardening exponent.

\subsection{Strain Hardening}
The true stress-true strain curve after yielding can often be approximated by:
\begin{equation}
\sigma = K \epsilon^n
\end{equation}
where $\sigma$ is the true stress, $\epsilon$ is the true strain, $K$ is the strength coefficient, and $n$ is the strain hardening exponent.

\subsection{Plastic Collapse Load}
For structures, the plastic collapse load can be determined by:
\begin{equation}
P_p = \sum (A_i \sigma_y)
\end{equation}
where $P_p$ is the plastic collapse load, $A_i$ are areas of cross-section contributing to plastic deformation, and $\sigma_y$ is the yield stress.

\begin{thebibliography}{9}

\bibitem{gere_goodno}
Gere, J. M., \& Goodno, B. J. (2012). \textit{Mechanics of Materials}. Cengage Learning.

\bibitem{hibbeler}
Hibbeler, R. C. (2020). \textit{Mechanics of Materials}. Pearson.

\bibitem{timoshenko}
Timoshenko, S. (1955). \textit{Strength of Materials, Part I \& II}. D. Van Nostrand Company.

\bibitem{beer_johnston}
Beer, F. P., Johnston, E. R., DeWolf, J. T., \& Mazurek, D. F. (2015). \textit{Mechanics of Materials}. McGraw-Hill Education.

\bibitem{MIT3.11}
MIT OpenCourseWare. Mechanics of Materials (3.11). 
\\\texttt{https://ocw.mit.edu/courses/3-11-mechanics-of-materials-fall-1999/}

\documentclass[12pt]{article}
\usepackage{amsmath}
\usepackage{amssymb}
\usepackage{booktabs}
\usepackage{hyperref}
\hypersetup{
    colorlinks=true,
    linkcolor=blue,
    filecolor=magenta,      
    urlcolor=cyan,
}
\title{Mechanics of Materials: Key Formulas, Notations, and Units}
\author{Viggo K. Hansen}
\date{\today}

\begin{document}

\maketitle

\tableofcontents

\section{\hyperref[sec:notation]{Notation and Units}}
\label{sec:notation}
\begin{tabular}{ll}
    \toprule
    Symbol & Description [Units] \\
    \midrule
    $P$ & Axial load [N] \\
    $A$ & Cross-sectional area [m$^2$] \\
    $L$ & Length of the member [m] \\
    $E$ & Young's modulus [Pa] \\
    $G$ & Shear modulus [Pa] \\
    $V$ & Shear force [N] \\
    $T$ & Torque [Nm] \\
    $J$ & Polar moment of inertia [m$^4$] \\
    $M$ & Bending moment [Nm] \\
    $I$ & Area moment of inertia [m$^4$] \\
    $\gamma$ & Shear strain [dimensionless] \\
    $\theta$ & Angle of twist [rad] \\
    $y$ & Distance from neutral axis [m] \\
    $\sigma_x, \sigma_y$ & Normal stresses in $x$ and $y$ directions [Pa] \\
    $\tau_{xy}$ & Shear stress in $xy$ plane [Pa] \\
    $\nu$ & Poisson's ratio [dimensionless] \\
    $\alpha$ & Coefficient of thermal expansion [K$^{-1}$] \\
    $\sigma_z$ & Normal stress in $z$ direction [Pa] \\
    \bottomrule
\end{tabular}

\section{Stress and Strain}
\subsection{Normal Stress}
\begin{equation}
\sigma = \frac{P}{A} \label{eq:normalstress}
\end{equation}
This equation \eqref{eq:normalstress} relates the normal stress to the applied axial load and the cross-sectional area.

\subsection{Shear Stress}
\begin{equation}
\tau = \frac{V}{A} \label{eq:shearstress}
\end{equation}
Shear stress \eqref{eq:shearstress} is calculated as the shear force divided by the area over which it acts.

\subsection{Normal Strain}
\begin{equation}
\varepsilon = \frac{\Delta L}{L} \label{eq:normalstrain}
\end{equation}
Normal strain \eqref{eq:normalstrain} is the ratio of change in length to the original length.

\section{Stress-Strain Relationships}
\subsection{Hooke's Law}
\begin{equation}
\sigma = E \varepsilon \label{eq:hooke}
\end{equation}
This linear relationship \eqref{eq:hooke} describes how stress is proportional to strain within the elastic limit. It assumes linear elastic behavior where the material returns to its original shape once the load is removed.

\subsection{Shear Modulus}
\begin{equation}
\tau = G \gamma \label{eq:shearmodulus}
\end{equation}
Shear stress \eqref{eq:shearmodulus} is directly proportional to shear strain, with $G$ as the proportionality constant.

\subsection{Generalized Hooke's Law}
\begin{align}
\varepsilon_x &= \frac{1}{E} \left( \sigma_x - \nu(\sigma_y + \sigma_z) \right) \label{eq:generalizedhooke1} \\
\gamma_{xy} &= \frac{\tau_{xy}}{G} \label{eq:generalizedhooke2}
\end{align}
These equations \eqref{eq:generalizedhooke1}, \eqref{eq:generalizedhooke2} account for the effect of Poisson's ratio in multi-axial stress states.

\section{Torsion}
\subsection{Torsional Stress}
\begin{equation}
\tau = \frac{T r}{J} \label{eq:torsionalstress}
\end{equation}
Here, $r$ is the radial distance from the center of the shaft. Equation \eqref{eq:torsionalstress} shows how stress varies across the cross-section of a shaft under torsion.

\subsection{Angle of Twist}
\begin{equation}
\theta = \frac{T L}{G J} \label{eq:angleoftwist}
\end{equation}
The angle of twist \eqref{eq:angleoftwist} is proportional to the torque and length, inversely to the shear modulus and polar moment of inertia.

\subsection{Thin-Walled Tubes (Average Shear Stress)}
\begin{equation}
\tau_{\text{avg}} = \frac{T}{2t A_m} \label{eq:thinwalledtube}
\end{equation}
where $t$ is the thickness of the tube wall and $A_m$ is the mean area enclosed by the centerline of the tube's cross-section. The formula \eqref{eq:thinwalledtube} assumes a uniform distribution of stress along the wall.

\subsection{Power in Torsion}
\begin{equation}
P = T \omega = 2\pi f T \label{eq:powerintorsion}
\end{equation}
This equation \eqref{eq:powerintorsion} shows how power is related to torque and rotational speed or frequency.

% ... (Continue with the rest of the sections, implementing the same format for equations, adding brief explanations, examples, assumptions, and limitations)

\section{Examples}
- **Normal Stress Example:** Consider a steel rod with a cross-sectional area of 10 cm² subjected to a force of 5000 N. Using \eqref{eq:normalstress}, the stress would be $\sigma = \frac{5000 \text{ N}}{0.001 \text{ m}^2} = 5,000,000 \text{ Pa}$ or 5 MPa.

- **Hooke's Law Example:** If a rubber band stretches by 5% under 100 N of force, using \eqref{eq:hooke}, you can calculate the modulus of elasticity or predict elongation under different forces.

% Continues for each relevant equation

\section{Assumptions and Limitations}
- **Normal Stress:** Assumes uniform stress distribution, which isn't accurate near edges or at points of load application.
- **Hooke's Law:** Valid only within the elastic limit; beyond this, plastic deformation occurs.

% Continues for each relevant equation

\section{Figures and Diagrams}
% Here, include a diagram showing stress distribution in a beam under bending
% \includegraphics[width=0.8\textwidth]{stress_distribution_beam.png}

% Diagram showing different buckling modes of a column
% \includegraphics[width=0.8\textwidth]{column_buckling_modes.png}

% ... (Continue with bibliography and end document)

\end{document}