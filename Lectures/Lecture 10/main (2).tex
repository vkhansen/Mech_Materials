\documentclass{beamer}
\usetheme{default}
\usepackage{amsmath, amssymb}
\usepackage{graphicx}
\usepackage{hyperref}
\usepackage{xcolor} % For colored text

\title{Week 10 Lecture Notes: Beam Deflections Using Double Integration}
\subtitle{Mechanics of Materials}
\author{Instructor: Viggo Hansen}
\date{March 19, 2025}

\begin{document}

% Title Slide
\begin{frame}
    \titlepage
\end{frame}

% Outline Slide
\begin{frame}{Outline}
    \tableofcontents
\end{frame}

% Section 1: Overview
\section{Overview}
\begin{frame}{Week 10 Overview: Beam Deflections}
    \begin{itemize}
        \item Topic: Analyzing beam deflections with the \textbf{double integration method}.
        \item Goals:
        \begin{itemize}
            \item Understand \textit{elastic curve}, \textit{slope}, and \textit{displacement}.
            \item Solve common beam problems step-by-step.
        \end{itemize}
        \item Why it matters: Ensures safe design of structures like bridges and buildings.
    \end{itemize}
\end{frame}

% Section 2: Introduction
\section{Introduction}
\begin{frame}{Introduction}
    \begin{itemize}
        \item Double integration: Key method to find how beams bend under loads.
        \item Outputs: Elastic curve (shape), slope (angle), displacement (movement).
        \item \textbf{Sign Convention}: Positive \( y \) is upward; downward deflections are negative.
    \end{itemize}
\end{frame}

% Section 3: Theoretical Background
\section{Theoretical Background}
\begin{frame}{Theoretical Background}
    \begin{itemize}
        \item Euler-Bernoulli beam theory:
        \[
        EI \frac{d^2y}{dx^2} = M(x)
        \]
        \item \( y \): deflection, \( x \): position, \( M(x) \): bending moment.
        \item \( E \): modulus of elasticity, \( I \): moment of inertia (second moment of area).
        \item Double integration:
        \begin{enumerate}
            \item \( \frac{dy}{dx} \) (slope) from first integral.
            \item \( y(x) \) (deflection) from second integral.
        \end{enumerate}
    \end{itemize}
\end{frame}

% Section 4: Steps of the Method
\section{Steps of the Method}
\begin{frame}{Steps of Double Integration}
    \begin{enumerate}
        \item Define \( M(x) \) (watch for \textcolor{red}{piecewise regions}!).
        \item Integrate: \( EI \frac{dy}{dx} = \int M(x) \, dx + C_1 \).
        \item Integrate again: \( EI y = \iint M(x) \, dx + C_1 x + C_2 \).
        \item Apply boundary conditions:
        \begin{itemize}
            \item Pinned/Roller: \( y = 0 \), slope \( \frac{dy}{dx} \neq 0 \).
            \item Fixed: \( y = 0 \), \( \frac{dy}{dx} = 0 \).
        \end{itemize}
        \item Constants \( C_1 \), \( C_2 \) depend on supports!
    \end{enumerate}
\end{frame}

% Section 5: Examples
\section{Examples}
\subsection{Simply Supported Beam with Central Load}
\begin{frame}{Example 1: Simply Supported Beam with Central Load}
    \begin{itemize}
        \item Setup: Length \( L \), load \( P \) at \( x = L/2 \).
        \item \textcolor{red}{Piecewise Moment}:
        \[
        M(x) = \begin{cases} 
        \frac{Px}{2}, & 0 \leq x \leq \frac{L}{2} \\
        \frac{P(L-x)}{2}, & \frac{L}{2} < x \leq L 
        \end{cases}
        \]
        \item Equation: \( EI \frac{d^2y}{dx^2} = M(x) \).
    \end{itemize}
    \begin{figure}
        \centering
        % Placeholder: https://www.thestructuralengineer.info/education/professional-examinations-preparation/calculation-examples/calculation-example-calculate-the-equation-of-the-elastic-curve
        \caption{Elastic curve (Source: thestructuralengineer.info)}
    \end{figure}
\end{frame}

\begin{frame}{Example 1: Solution}
    \begin{itemize}
        \item For \( 0 \leq x \leq L/2 \):
        \item First integration: \( EI \frac{dy}{dx} = \frac{Px^2}{4} + C_1 \).
        \item Second integration: \( EI y = \frac{Px^3}{12} + C_1 x + C_2 \).
        \item Boundaries: \( y(0) = 0 \) (pin), \( y(L) = 0 \) (roller).
        \item Solve: \( C_2 = 0 \), \( C_1 = -\frac{PL^2}{16} \) (using symmetry or full span).
        \item Max deflection: \( \delta_{\text{max}} = -\frac{PL^3}{48EI} \) at \( x = L/2 \).
    \end{itemize}
    See video: \href{https://www.youtube.com/watch?v=EWhL-mixfaI}{youtube.com/watch?v=EWhL-mixfaI}
\end{frame}

\subsection{Cantilever Beam with End Load}
\begin{frame}{Example 2: Cantilever Beam with End Load}
    \begin{itemize}
        \item Setup: Length \( L \), load \( P \) at free end (\( x = L \)).
        \item Moment: \( M(x) = P(L - x) \) (single region).
        \item Equation: \( EI \frac{d^2y}{dx^2} = P(L - x) \).
    \end{itemize}
    \begin{figure}
        \centering
        % Placeholder: https://www.thestructuralengineer.info/education/professional-examinations-preparation/calculation-examples/calculation-example-calculate-the-equation-of-the-elastic-curve
        \caption{Cantilever deflection (Source: thestructuralengineer.info)}
    \end{figure}
\end{frame}

\begin{frame}{Example 2: Solution}
    \begin{itemize}
        \item First integration: \( EI \frac{dy}{dx} = P(Lx - \frac{x^2}{2}) + C_1 \).
        \item Second integration: \( EI y = P(\frac{Lx^2}{2} - \frac{x^3}{6}) + C_1 x + C_2 \).
        \item Boundaries: \( y(0) = 0 \), \( \frac{dy}{dx}(0) = 0 \) (fixed end).
        \item Solve: \( C_1 = 0 \), \( C_2 = 0 \).
        \item Deflection at end: \( y(L) = -\frac{PL^3}{3EI} \) (downward).
    \end{itemize}
    See video: \href{https://www.youtube.com/watch?v=MJxIjG-32JA}{youtube.com/watch?v=MJxIjG-32JA}
\end{frame}

\subsection{Simply Supported Beam with Uniform Load}
\begin{frame}{Example 3: Simply Supported Beam with Uniform Load}
    \begin{itemize}
        \item Setup: Length \( L \), uniform load \( w \).
        \item Moment: \( M(x) = \frac{wx}{2}(L - x) \) (single region).
        \item Equation: \( EI \frac{d^2y}{dx^2} = \frac{wx}{2}(L - x) \).
    \end{itemize}
    \begin{figure}
        \centering
        % Placeholder: https://www.thestructuralengineer.info/education/professional-examinations-preparation/calculation-examples/calculation-example-calculate-the-equation-of-the-elastic-curve
        \caption{Uniform load deflection (Source: thestructuralengineer.info)}
    \end{figure}
\end{frame}

\begin{frame}{Example 3: Solution}
    \begin{itemize}
        \item First integration: \( EI \frac{dy}{dx} = \frac{w}{2}(\frac{Lx^2}{2} - \frac{x^3}{3}) + C_1 \).
        \item Second integration: \( EI y = \frac{w}{2}(\frac{Lx^3}{6} - \frac{x^4}{12}) + C_1 x + C_2 \).
        \item Boundaries: \( y(0) = 0 \) (pin), \( y(L) = 0 \) (roller).
        \item Solve: \( C_2 = 0 \), \( C_1 = -\frac{wL^3}{24} \).
        \item Max deflection: \( \delta_{\text{max}} = -\frac{5wL^4}{384EI} \) at \( x = L/2 \).
    \end{itemize}
    See video: \href{https://www.youtube.com/watch?v=6l5pjdlAtlc}{youtube.com/watch?v=6l5pjdlAtlc}
\end{frame}

% Section 6: Practical Tips
\section{Practical Tips}
\begin{frame}{Practical Tips for Solving Problems}
    \begin{itemize}
        \item Sketch beam, loads, supports—label everything!
        \item Check \( M(x) \) with equilibrium (forces/moments balance).
        \item \textbf{Boundary Conditions}:
        \begin{itemize}
            \item Pinned/Roller: \( y = 0 \), slope can vary.
            \item Fixed: \( y = 0 \), \( \frac{dy}{dx} = 0 \).
        \end{itemize}
        \item Watch \textcolor{red}{piecewise \( M(x) \)}—define regions clearly.
        \item Units: \( E \) (Pa), \( I \) (m\(^4\)), \( M \) (N·m).
    \end{itemize}
\end{frame}

% Section 7: Resources
\section{Resources}
\begin{frame}{Resources for Week 10}
    \begin{itemize}
        \item Textbook: Beam deflection chapter.
        \item Figures: \url{https://www.thestructuralengineer.info/education/professional-examinations-preparation/calculation-examples/calculation-example-calculate-the-equation-of-the-elastic-curve}.
        \item Videos:
        \begin{itemize}
            \item Ex 1: \href{https://www.youtube.com/watch?v=EWhL-mixfaI}{youtube.com/watch?v=EWhL-mixfaI}.
            \item Ex 2: \href{https://www.youtube.com/watch?v=MJxIjG-32JA}{youtube.com/watch?v=MJxIjG-32JA}.
            \item Ex 3: \href{https://www.youtube.com/watch?v=6l5pjdlAtlc}{youtube.com/watch?v=6l5pjdlAtlc}.
        \end{itemize}
        \item Office hours: Bring questions!
    \end{itemize}
\end{frame}

% Section 8: Conclusion
\section{Conclusion}
\begin{frame}{Conclusion}
    \begin{itemize}
        \item Double integration: Step-by-step tool for beam deflections.
        \item Classic examples: Simply supported (point/uniform load), cantilever.
        \item Key for safe structural design.
        \item Next week: Macaulay’s method for complex loads.
    \end{itemize}
\end{frame}

\end{document}