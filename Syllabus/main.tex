\documentclass[12pt]{article}

\usepackage[utf8]{inputenc}
\usepackage{geometry}
\geometry{a4paper, margin=1in}
\usepackage{hyperref}
\hypersetup{
    colorlinks=true,
    linkcolor=blue,
    filecolor=magenta,
    urlcolor=cyan,
}
\usepackage{graphicx}
\usepackage{doi}
\usepackage{fancyhdr}
\usepackage{xcolor}
\usepackage{booktabs}
\usepackage{tabularx}

\pagestyle{fancy}
\fancyhf{}
\lhead{Mechanics of Materials: Course Syllabus}
\rhead{Version 1.1}
\cfoot{\thepage}

\title{\textbf{Mechanics of Materials: Course Syllabus} \\ \small{Version 1.1, Revised: \today}}
\author{Lecturer: Viggo Hansen\\
International School of Engineering (ISE)}
\date{Academic Year: 2024 \\ Semester: Second}

\begin{document}

\maketitle

% Hyperlinked Table of Contents
{
\hypersetup{linkcolor=black} % Change link color for TOC
\tableofcontents
}

% Define document control information
\newcommand{\documentversion}{Version 1.1}
\newcommand{\documentdate}{\today}

% Setup headers and footers
\pagestyle{fancy}
\fancyhf{}
\fancyhead[L]{\documentversion}
\fancyhead[R]{\documentdate}
\fancyfoot[C]{\thepage\ of \pageref{LastPage}} % This line sets up the page numbering
\renewcommand{\headrulewidth}{0.4pt}
\renewcommand{\footrulewidth}{0.4pt}

\begin{enumerate}
    \item \textbf{Course Number:} 2183213
    \item \textbf{Course Credit:} 3 (3-0-6)
    \item \textbf{Course Title:} Mechanics of Materials
    \item \textbf{Faculty/Department:} International School of Engineering
    \item \textbf{Semester:} Second
    \item \textbf{Academic Year:} 2024
    \item \textbf{Instructors:}
        \begin{itemize}
            \item Section 1: Asst. Prof. Tawan Paphapote
            \item Section 2: Viggo Hansen
        \end{itemize}
    \item \textbf{Conditions:}
        \begin{itemize}
            \item Prerequisite: 2183212
            \item Corequisite: None
            \item Concurrent: None
        \end{itemize}
    \item \textbf{Status:} Required
    \item \textbf{Curriculum:}
        \begin{itemize}
            \item B.E. in Automotive Design and Manufacturing Engineering
            \item B.E. in Robotics and AI
        \end{itemize}
    \item \textbf{Degree:} Bachelor's Degree
    \item \textbf{Total Number of Hours:} 45 hours
    \item \textbf{Course Description:} \\
    Concepts of stress and strain; stress and strain components; plane stress and plane strain; Mohr's circle for plane stress; Hooke's law and modulus of elasticity; engineering stress-strain diagrams; working stress; factor of safety; axial loading including statically indeterminate problems and temperature changes; thin-walled pressure vessels; torsion of circular shafts; statically indeterminate shafts; beam analysis including stress, deflection; Euler's formula for buckling; combined stress; theories of failure.

    \item \textbf{Course Outline:}
        \begin{itemize}
            \item \textbf{Objectives:}
                \begin{enumerate}
                    \item Determine stress and strain in simple mechanical components.
                    \item Explain material behaviors based on mechanical properties.
                    \item Analyze plane stress at a point, construct and apply Mohr's circle.
                    \item Determine stress and deformation under various load types including axial, torsional, flexural, pressure vessel, and combined loading.
                    \item Solve statically indeterminate problems using additional deformation equations.
                    \item Calculate beam deflection including elastic curve.
                    \item Determine buckling loads for columns with various boundary conditions.
                \end{enumerate}
        \end{itemize}

        \textbf{Course Content:}

\begin{tabularx}{\textwidth}{>{\raggedright\arraybackslash}p{1.5cm}>{\raggedright\arraybackslash}Xp{2cm}}
    \toprule
    \textbf{Week} & \textbf{Topics} & \textbf{Remarks} \\
    \midrule
    1 & Introduction, External and internal loads, Normal and Shear stress, Allowable stress, Deformation, Strain, Tensile/Compressive tests, Stress-strain behavior, Ductile vs. Brittle materials, Hooke's law, Poisson's ratio & Chapter 1 \\
    2 & Plane stress, Stress transformation equations, Principal stress, Maximum in-plane shear stress, Mohr's circle, Absolute maximum shear stress & Chapter 2 \\
    3 & Deformation under axial load, Thermal stress & Chapter 3 \\
    4 & Statically Indeterminate Members & Chapter 4 \\
    5 & Torsion, Torsion formula & Chapter 5 \\
    6 & Power transmission, Angle of twist, Statically indeterminate torque-loaded members & Chapter 6 \\
    7 & Shear and Moment Diagrams & Chapter 7 \\
    8 & Shear and Moment Diagrams (Graphical Method), Geometric properties (centroid, Q, I) & Chapter 8 \\
    9 & Bending, Flexural formula, Shear in beams, Shear formula & Chapter 9 \\
    10 & Elastic curve, Slope and displacement by integration & Chapter 10 \\
    11 & Statically Indeterminate Beams by Integration & Chapter 11 \\
    12 & Thin-Walled Pressure Vessels & Chapter 12 \\
    13 & Combined Loadings & Chapter 13 \\
    14 & Buckling, Critical load, Columns with various supports & Chapter 14 \\
    \bottomrule
\end{tabularx}


\subsection*{Homework Notes:}
\begin{itemize}
    \item \textbf{Contributions:} Homework can be submitted as:
        \begin{itemize}
            \item Python functions relevant to coursework (optimizations encouraged)
            \item Jupyter Notebooks for textbook problems or student/instructor-created problems
            \item Corrections to existing software or content
        \end{itemize}
        Assignments may be submitted as PDF or Word documents.
    \item \textbf{Submission:} Use the Course GitHub Repository for code (MATLAB, Python, etc.).
    \item \textbf{GitHub Access:} Email your GitHub username to \href{mailto:vkhansen@eng.chula.ac.th}{vkhansen@eng.chula.ac.th} for access.
\end{itemize}

\textbf{Method of Teaching:} Lectures \\
\textbf{Teaching Media:} Class notes, GitHub Source Code \\
\textbf{Assignment through Network System:}
\begin{itemize}
    \item Method: MyCourseville
    \item LMS: MyCourseville
\end{itemize}

\begin{itemize}
    \item \textbf{Google Groups Mailing List:} Google Groups for discussions, updates, and Q\&A:
        \begin{itemize}
            \item \href{https://groups.google.com/g/mechanics-of-materials-2183213}{Admin Panel: mechanics-of-materials-2183213}
            \item \href{mailto:mechanics-of-materials-2183213@googlegroups.com}{Email: mechanics-of-materials-2183213@googlegroups.com}
        \end{itemize}
    \item \textbf{GitHub Repository:} To submit HW/Source code:
        \begin{itemize}
            \item Direct link to \href{https://github.com/vkhansen/Mech_Materials}{Course GitHub Repo}
            \item Open repository through \href{https://desktop.github.com/}{GitHub Desktop App}
            \item For repository write access, email GitHub username to: \href{mailto:vkhansen@eng.chula.ac.th}{vkhansen@eng.chula.ac.th}
        \end{itemize}
    \item \textbf{Notebook LM:} To access and interact with the course materials in Google Notebook LM:
        \begin{itemize}
            \item Follow direct link: \href{https://notebooklm.google.com/notebook/eb161394-bf08-4333-b72a-5e132e2746ff}{Google Notebook LM}.
            \item For access email: \href{mailto:vkhansen@eng.chula.ac.th}{vkhansen@eng.chula.ac.th}
        \end{itemize}
\end{itemize}

\section*{Evaluation}
\begin{itemize}
    \item Weekly Quizzes (Paper-Based, 1 Note Page Allowed) based on topics from the Course Outline: 30\%
    \item Homework Assigned Weekly via MCV (Submit as source code, PDF, to GitHub Repo): 20\%
    \item Midterm: 20\%
    \item Final Exam: 30\%
\end{itemize}

\textbf{Reading Lists:}
\begin{itemize}
    \item Required Textbook: Hibbeler, R.C., \textit{Mechanics of Materials}, Prentice Hall.
    \item Course Notebook LM: \href{https://notebooklm.google.com/notebook/eb161394-bf08-4333-b72a-5e132e2746ff}{Google Notebook LM}
\end{itemize}

\textbf{Teacher Evaluation:}
\begin{itemize}
    \item Standard Chulalongkorn University Evaluation form: Lecture course (Form 04)
    \item Continuous improvement through Program Outcomes
    \item Discussions aimed at enhancing CU graduate qualifications.
\end{itemize}
\end{enumerate}

\section*{Document History}
\begin{itemize}
    \item Version 1.0 - 01/26/2024 - Initial draft by Viggo Hansen
    \item Version 1.1 - \today - Revised by Viggo Hansen
\end{itemize}
\label{LastPage}
\end{document}