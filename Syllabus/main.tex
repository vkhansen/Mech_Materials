\documentclass{article}
\usepackage[utf8]{inputenc}
\usepackage{amsmath}
\usepackage{tabularx}
\usepackage{booktabs}
\usepackage{geometry}
\usepackage{fancyhdr}
\usepackage{lastpage}
\geometry{margin=1in}

\title{Mechanics of Materials: Course Syllabus}
\author{Viggo K. Hansen}
\date{January 18, 2025}

% Define document control information
\newcommand{\documentversion}{Version 1.0}
\newcommand{\documentdate}{\today}

% Setup headers and footers
\pagestyle{fancy}
\fancyhf{}
\fancyhead[L]{\documentversion}
\fancyhead[R]{\documentdate}
\fancyfoot[C]{\thepage\ of \pageref{LastPage}}
\renewcommand{\headrulewidth}{0.4pt}
\renewcommand{\footrulewidth}{0.4pt}

\begin{document}

\maketitle

\begin{enumerate}
    \item \textbf{Course Number:} 2183213
    \item \textbf{Course Credit:} 3 (3-0-6)
    \item \textbf{Course Title:} Mechanics of Materials
    \item \textbf{Faculty/Department:} International School of Engineering
    \item \textbf{Semester:} First
    \item \textbf{Academic Year:} 2024
    \item \textbf{Instructors:}
        \begin{itemize}
            \item Section 1: Asst. Prof. Tawan Paphapote
            \item Section 2: Viggo Hansen
        \end{itemize}
        \textbf{Office Hours:} TBA, Online only
    \item \textbf{Conditions:}
        \begin{itemize}
            \item Prerequisite: 2183212
            \item Corequisite: None
            \item Concurrent: None
        \end{itemize}
    \item \textbf{Status:} Required
    \item \textbf{Curriculum:}
        \begin{itemize}
            \item B.E. in Automotive Design and Manufacturing Engineering
            \item B.E. in Robotics and AI
        \end{itemize}
    \item \textbf{Degree:} Bachelor's Degree
    \item \textbf{Total Number of Hours:} 45 hours
    \item \textbf{Course Description:} \\
    Concept of stress and strain; stress and strain components; plane stress and plane strain; Mohr's circle of plane stress; Hooke's law and modulus of elasticity; engineering stress-strain diagrams; working stress; factor of safety; axial loading including statically indeterminate problems and temperature changes; thin-walled pressure vessels; torsion of circular shafts; statically indeterminate shafts; beam analysis including stress, deflection; Euler's formula for buckling; combined stress; theories of failure.
    
    \item \textbf{Course Outline:}
        \begin{itemize}
            \item \textbf{Objectives:}
                \begin{enumerate}
                    \item Determine stress and strain in simple mechanical components.
                    \item Explain material behaviors based on mechanical properties.
                    \item Analyze plane stress at a point, construct and apply Mohr's circle.
                    \item Determine stress and deformation under various load types including axial, torsional, flexural, pressure vessel, and combined loading.
                    \item Solve statically indeterminate problems using additional deformation equations.
                    \item Calculate beam deflection including elastic curve.
                    \item Determine buckling loads for columns with various boundary conditions.
                \end{enumerate}
        \end{itemize}

        \textbf{Course Content:}

\begin{tabularx}{\textwidth}{lXl}
    \toprule
    \textbf{Period} & \textbf{Topics} & \textbf{Remarks} \\
    \midrule
    1 & Introduction, External and internal loads, Normal and Shear stress, Allowable stress, Deformation, Strain, Tensile/Compressive tests, Stress-strain behavior, Ductile vs. Brittle materials, Hooke's law, Poisson's ratio & Chapter 1 \\
    2 & Plane stress, Stress transformation equations, Principal stress, Maximum in-plane shear stress, Mohr's circle, Absolute maximum shear stress & Chapter 2 \\
    3 & Deformation under axial load, Thermal stress & Chapter 3 \\
    4 & Statically Indeterminate Members & Chapter 4 \\
    5 & Torsion, Torsion formula & Chapter 5 \\
    6 & Power transmission, Angle of twist, Statically indeterminate torque-loaded members & Chapter 6 \\
    7 & Shear and Moment Diagrams & Chapter 7 \\
    8 & Shear and Moment Diagrams (Graphical Method), Geometric properties (centroid, Q, I) & Chapter 8 \\
    9 & Bending, Flexural formula, Shear in beams, Shear formula & Chapter 9 \\
    10 & Elastic curve, Slope and displacement by integration & Chapter 10 \\
    11 & Statically Indeterminate Beams by Integration & Chapter 11 \\
    12 & Thin-Walled Pressure Vessels & Chapter 12 \\
    13 & Combined Loadings & Chapter 13 \\
    14 & Buckling, Critical load, Columns with various supports & Chapter 14 \\
    \bottomrule
\end{tabularx}

\subsection*{Homework Notes:}
\begin{itemize}
    \item \textbf{Contributions:} Homework can be submitted as:
        \begin{itemize}
            \item Python functions relevant to coursework (optimizations encouraged)
            \item Jupyter Notebooks for textbook problems or student/instructor-created problems
            \item Corrections to existing software or content
        \end{itemize}
        Assignments can also be submitted as PDF or Word documents.
    \item \textbf{Submission:} Use the Course GitHub Repository for code (MATLAB, Python, etc.). Refer to Code-First Learning slides for guidelines.
    \item \textbf{GitHub Access:} Email your GitHub username to \href{mailto:vkhansen@eng.chula.ac.th}{vkhansen@eng.chula.ac.th} for access.
\end{itemize}

        \textbf{Method of Teaching:} Lectures, Code-first \\
        \textbf{Teaching Media:} Class notes, Github Source Code \\
        \textbf{Assignment through Network System:}
        \begin{itemize}
            \item Method: MyCourseville, Github
            \item LMS: MyCourseville, Github
        \end{itemize}

\section*{Evaluation}
\begin{itemize}
    \item Weekly Quizzes (Paper-Based, 1 Note Page Allowed) based on topics from the Course Outline: 30\%
    \item Homework Assigned Weekly via MCV (Submit as source code, PDF, to GitHub Repo): 20\%
    \item Midterm: 20\%
    \item Final Exam: 30\%
\end{itemize}

\textbf{Reading Lists:}
\begin{itemize}
    \item Required Textbook: Hibbeler, R.C., \textit{Mechanics of Materials}, Prentice Hall.
\end{itemize}

\textbf{Teacher Evaluation:}
\begin{itemize}
    \item Standard Chulalongkorn University Evaluation form: Lecture course (Form 04)
    \item Continuous improvement through Program Outcomes
    \item Discussions aimed at enhancing CU graduate qualifications.
\end{itemize}
\end{enumerate}

\section*{Document History}
\begin{itemize}
    \item Version 1.0 - 01/26/2024 - Initial draft by Viggo Hansen
\end{itemize}

\end{document}